\documentclass{beamer}

\usetheme{Berkeley}

\numberwithin{equation}{section} % Number equations with sections

\usepackage{amsmath}
\usepackage{amsfonts}
\usepackage[latin1]{inputenc}
\usepackage{amsmath}
\usepackage{amsfonts}
\usepackage{amssymb}
\usepackage{makeidx}
\usepackage{tabularx}
\usepackage{url}
\usepackage{cite}
\usepackage{hyperref}
\usepackage{multicol}
\usepackage{fancyvrb}
\usepackage{color}
\usepackage{subfigure}
\usepackage{graphicx}
\usepackage{pdfpages}
% use \usepackage[pdfborder=0in]{hyperref} instead to disable red box links
%\usepackage[T1]{fontenc}

\title{Using a Neural Network to Solve the NP-Complete Subset Sum Problem}
\author{Aaron Meurer}
\date{November 17, 2011}

\begin{document}

\begin{frame}
    \titlepage
\end{frame}

\begin{frame} 
    \frametitle{Agenda} 
    \tableofcontents 
\end{frame} 

\section{Introduction to the Problem}

\begin{frame}
    \frametitle{Subset Sum Problem}
    \begin{itemize}
        \item \textbf{Problem:} Given a finite set of positive and negative integers, determine if there exists a subset that sums to zero.
        \pause
        \item Formally, given a finite set $A \subset \mathbb{Z}_+ \cup \mathbb{Z}_-$, determine if there exists $B \subseteq A$ such that $\sum_{x\in B}x = 0$.
        \pause
       \item \textbf{Example:} $A=\{1, 5, -14, -6, 3\}$ satisfies the property, with $B=\{1, 5, -6\}$ because $1 + 5 + -6 = 0$.
       \pause
       \item $A=\{2, 6, 8, -7\}$ does not satisfy the property (no subset sums to 0).
    \end{itemize}
\end{frame}

\begin{frame}
    \frametitle{Subset Sum Problem}
    \begin{itemize}
        \item The subset sum problem is NP-Complete.
        \pause
        \item In general terms, it means that no one knows how to solve the problem efficiently.
        \pause
        \item The brute force approach requires checking all $2^n - 1$ subsets of a set of size $n$, which, due the exponential growth with respect to $n$, become unpractical for all but small $n$.
    \end{itemize}
\end{frame}

\section{Neural Network Implementation}

\begin{frame}
    \frametitle{Neural Network Implementation}
    \begin{itemize}
        \item I implemented the multilayer backpropagation neural network algorithm from scratch in Python.
        \pause
        \item This allowed me to tweak things with great precision.
    \end{itemize}
\end{frame}

\begin{frame}
    \frametitle{Parameters}
    \begin{itemize}
        \item Start with random weights from -1 to 1.
        \pause
        \item Each layer in the network has a bias node, which does not receive input from the previous layer.
        \pause
        \item Initialize bias node weights to 1.
        \pause
        \item Train the bias nodes with the same backpropagation rule as the other nodes.
    \end{itemize}
\end{frame}

\begin{frame}
    \frametitle{Architecture}
    \begin{itemize}
        \item I settled on using four layers of nodes.
        \pause
        \item The first three layers have $n$ nodes, and the last layer has one node (for binary output).
        \pause
        \item I tested for $n\in\{10, 20, \ldots, 80\}$.
    \end{itemize}
\end{frame}

\section{Questions}

\begin{frame}
    \frametitle{Questions?}
    \huge{Questions?}
\end{frame}

\end{document}
